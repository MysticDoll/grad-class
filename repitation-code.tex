
\documentclass[12pt,a4paper]{article}
\usepackage[truedimen,margin=25truemm]{geometry}
\usepackage{amsmath}
\usepackage{hyperref}
\begin{document}
\subsection*{\texorpdfstring{${}^\forall q, n, F = F_q$上の反復符号$\mathcal{R}_n$の誤り訂正能力}{{}^{}\forall q, nに対してF = F_q上の反復符号\mathcal{R}_nの誤り訂正能力}}\label{forall-q-nux306bux5bfeux3057ux3066f-f_qux4e0aux306eux30d1ux30eaux30c6ux30a3ux691cux67fbux7b26ux53f7mathcalp_nux306eux8aa4ux308aux8a02ux6b63ux80fdux529b}

最近傍復号を利用した場合に${}^\forall q,n$に対して、$F = F_q$上の反復符号$\mathcal{R}_n$で$w = \lfloor \frac{n-1}{2} \rfloor$個までの誤りが訂正できることを示す。

\subsection*{Proof}\label{proof}

送信語$\mathbf{u} = (u \dots u) \in \mathcal{R}_n \  (u \in F)$ を送信するとき
受信語$\mathbf{v}$は$w \leq \lfloor \frac{n-1}{2} \rfloor$個のシンボルが誤っているとする。

正しく転送されているシンボルの個数は$r = n - w$で、$d_H (\mathbf u , \mathbf v) = w = n - r$である。

送信語$\mathbf{u}$と異なる符号語$\mathcal{R}_n \ni \mathbf{u}^\prime = (u^\prime \dots u^\prime) \neq \mathbf{u}$について考える。このとき$r$個の符号語が正しく転送されているため、$d_H (\mathbf{u}^\prime, \mathbf{v}) \geq r$

$w = n - r \leq \lfloor \frac{n - 1}{2} \rfloor \Leftrightarrow 2(n - r) \leq n - 1 \Leftrightarrow n + 1 \leq 2r \Leftrightarrow \frac{n + 1}{2} \leq r$

よって$d_H(\mathbf{u} , \mathbf{v}) = w \leq \lfloor \frac{n - 1}{2} \rfloor < \frac{n + 1}{2} \leq r \leq d_H (\mathbf{u}^\prime , \mathbf{v})$

よって、他のどんな符号語よりも送信語のほうがハミング距離の意味で受信語に近いため、最近傍復号によって受信語に訂正されることがわかる。

\end{document}
