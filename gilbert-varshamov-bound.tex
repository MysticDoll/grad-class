\documentclass[12pt,a4paper]{article}
\usepackage[truedimen,margin=25truemm]{geometry}
\usepackage{amsmath}
\usepackage[dvipdfmx]{hyperref}
\usepackage{amssymb}

\begin{document}
  \subsection*{6.5 Gilbert-Varshamov限界}
    
    良い誤り訂正能力を維持しながら、伝達速度$R = \frac{1}{n} \log_q M$を最大化するために、与えられた$q, n$及び$t $ (または同値な$d$)に対して、可能な限り大きい値$M = \mid C \mid $となる符号を探すことは興味深い。
    $A_q (n, d)$を任意の符号長$n$、最小距離$d$の$q$元符号の符号語数の最大値と置く。ここで$d \leq n$である。
    ハミングの球充填限界式 (定理6.15)から$A_q (n, d)$の上界は次のように与えられる。

    \[A_q (n, d) \big(1 + \binom n1 (q - 1) + \binom n2 {(q - 1)}^2 + \cdots + \binom nt {(q - 1)}^t \big) \leq q^n\]

    ここで、定理6.10から$t = \lfloor (d - 1) / 2 \rfloor$である。

    \subsubsection*{例 6.20}
      
    $q = 2$と$d = 3$の場合、$t = 1$で、例6.16で見たように、$A_2 (n, 3) \leq \lfloor 2^n / (n + 1) \rfloor$である。
    よって、$n = 3, 4, 5, 6, 7, \dots $に対して、$A_2 (n, 3) = 2, 3, 5, 9, 16, \dots $である。

      \paragraph{練習 6.9}
        
      例6.20で$A_2 (n, 3)$の上界を求めたように、$A_3(n, 3)$の上界を求めよ。
      ハミングの球充填限界式は$A_2 (n, 4)$と$A_2 (n,4)$に関して何を意味するか?

    似たような議論から、与えられた$q, n$そして$d$に対して、与えられた最小の符号語数を持つ符号が存在することを示すことによって、$A_q(n, d)$の下界が得られる。
    これがGilbert-Varshamov 限界である。

    \subsubsection*{定理 6.21}
      $q \geq 2$かつ$n \geq d \geq 1$のとき

      \[A_q (n, d) \big( 1 + \binom n1 (q - 1) + \binom n2 {(q- 1)}^2 + \cdots + \binom{n}{(d - 1)}^{^{d -1} } \big) \geq q^n \]

      \paragraph{証明}
      与えられた$q, n$及び$d$を満たす全ての符号に関して、$C$を最大の符号語数を持つ符号と置く。つまり、$M = \mid C \mid = A_q (n, d)$である。
      $\mathbf{u} \in C$なる全ての球

      \[ S_{d - 1} (\mathbf{u}) = \lbrace \mathbf{v} \in \mathcal{V} \mid d(\mathbf{u}, \mathbf{v}) \leq d - 1 \rbrace \]

      は、$\mathcal{V}$を覆う。なぜなら、もし$\mathbf{v} \in \mathcal{V}$がどの$S_{d - 1} (\mathbf{u})$にも含まれないとすると、任意の$\mathbf{u} \in C$に対して$d (\mathbf{u}, \mathbf{v}) \geq d$で、
      符号$C^\prime = C \cup \lbrace \mathbf{v} \rbrace$は同じ$q, n$及び$d$の値を持つが、これは$C$の選び方に反するからである。
      (6.6) を証明した議論によって、それぞれ$M$個の球$S_{d-1}(\mathbf{u})$は$\sum_{i=0}^{d-1}\binom ni {(q - 1)}^i$個のベクトルを含んでいる。
      以上から、これらの球は全ての$\mathcal{V}$上$q^n$個の全てのベクトルを含んでいるのて、上式を満たす。

    \subsubsection*{例 6.22}
      もし、再び$q = 2$と$d = 3$をとると (とって$t = 1$である) 、定理 6.21は全ての$n \geq 3$に対して、
      
      \[ A_2 (n, 3) \big( 1 + n + \frac{n(n - 1)}{2} \big) \geq 2^n \]

      を与え、よって$A_2(n,3) \geq 2^{n+1} / (n^2 + n + 2) $である。$A_q (n, d)$は整数なので、

      \[A_2 (n, 3) \geq \lceil 2^{n + 1} / (n^2 + n + 2) \rceil \]

      である。$n = 3, 4, 5, 6, 7, \ldots$に対して、$A_2 (n, 3) \geq 2, 2, 2, 3, 5$である。
      例6.20で、上界と下界を比べる場合、$A_2(3, 3) = 2$である。例えば、2元反復符号$\mathcal{R}_3$はこの境界を満たす。
      $n = 4$のとき、$2 \leq A_2 (4, 3) \leq 3$で、$A_2(4, 3) = 2 \text{ or } 3$である。
      
      \paragraph{練習 6.10}
        $A_2 (4, 3) = 2$を示し、この境界に達する符号を示せ。

      \paragraph{練習 6.11}
        $A_3 (n, 3)$の下界を求めよ。

        多くの$q, n$と$d$に対して、$A_q (n, d)$の上界と下界には重要な差があり、この正確な値を求めるのは難しい問題だ。実際、多くの場合この値はわからない。
        場合によっては特殊な符号がこの値の存在を教えてくれる。$q = 2, d = 3$で$n = 7$のとき、ハミング符号$\mathcal{H}_7$は定理6.15から上界$M \leq 16$に達する。よって$A_2 (7, 6) = 16$である。
        より一般的に、\S 7.4で$n$が$2^c - 1$の形をしているとき、$A_2(n, 3)$は上界$2^{n - c}$に達することを確認する。

        2元符号の場合、定理6.21は以下の形になる。

        \[ A_2(n, d) \big( 1 + \binom n1 + \binom n2 + \cdots + \binom{n}{d - 1} \big) \geq 2^n \]

        \noindent 今、練習5.7から$Q < \frac 12$のとき

        \[ \underset{i \leq nQ}{\sum} \binom ni \leq 2^{nH_2 (Q)} \]

        \noindent よって、$d \leq \lfloor n / 2 \rfloor$に対して、

        \[ \log_2 A_2(n, d) \geq n(1 - H_2 (\frac{d - 1}{n})) \]

        \noindent なぜなら、2元符号は伝送速度$R = \frac 1n \log_2 M$で、これは$d \leq \lfloor n / 2 \rfloor$のとき符号長が$n$、最小距離が$d$で、伝送速度が

        \[ R \geq 1 - H_2 ( \frac{d - 1}{n}) \]

        \noindent なるような符号が存在することを示しているからである。

        これは定理6.10によって $t = \lfloor (d - 1) / 2 \rfloor$ の下 \S 6.4で証明したハミングの漸近的上界

        \[ R \leq 1 - H_2 (\frac tn ) \]

        \noindent と比べることが出来る。
        \noindent 図6.6は$R$のこれら2つの境界によって定義される領域を表している。

\end{document}
