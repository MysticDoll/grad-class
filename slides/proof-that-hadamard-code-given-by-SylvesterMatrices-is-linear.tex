\documentclass[dvipdfmx,10pt,jsarticle]{beamer}
\usetheme{CambridgeUS}

\usepackage{amsmath}
\usepackage{amssymb}
\usepackage{newtxtext}
\title{Proof of Linearity of Hadamard Code given by Sylvester Matrices}
\author{Mitsuru Takigahira}
\date[2017/10/01]{}

\begin{document}
  \frame{\maketitle}
  
  \begin{frame}{証明}
    数学的帰納法で示す。

    $n = 2^m$のとき
    \begin{itemize}
      \item Sylvester Matrixを$S^m$
      \item $S^m$から生成されるHadamard Codeを$C_m$
      \item $C_m$の各成分を$\mathbf{u}_1^m, \bar{\mathbf{u}}^m_1, \ldots , \mathbf{u}_{2^m}^m, \bar{\mathbf{u}}^m_{2^m}$
    \end{itemize}
    と表記することにする
  \end{frame}

  \begin{frame}{証明 (1/3)}
    \begin{block}{$m=0$のとき}
      $S^m = \begin{pmatrix}
        1
      \end{pmatrix} \text{or} \begin{pmatrix}
        -
      \end{pmatrix}$
      より、$C_0 = \lbrace \begin{pmatrix} 1 \end{pmatrix} \begin{pmatrix} 0 \end{pmatrix} \rbrace$ \\
      よって、$C_0$は$\begin{pmatrix} 1 \end{pmatrix}$を基底として線形になり、$C_0$は線形符号。
    \end{block}
    \begin{block}{帰納法の仮定}
      $m = k \geq 0$のとき$C_m$が線形になると仮定する。 \\
      このとき$C_m$は$2^{k+1}$個の符号$\lbrace \mathbf{u}^k_1, \bar{\mathbf{u}}^k_1, \ldots , \mathbf{u}^k_{2^m}, \bar{\mathbf{u}}^k_{2^m} \rbrace$を持ち、\\
      これらは$k + 1$個の基底$\lbrace \mathbf{e}^k_1, \ldots, \mathbf{e}^k_{k+1}\rbrace$のもと線形である。
    \end{block}
  \end{frame}

  \begin{frame}{証明 (2/3)}
    \begin{block}{$S^{k+1}$から作られる符号の線形性}
      補題6.24より$S^{k+1} = \begin{pmatrix}
        S^k & S^k \\ 
        S^k & - S^k
      \end{pmatrix}$
      なので、$C_{k+1}$に含まれる符号は\\
      $1 \leq i \leq 2^k$に対して、 $(\mathbf{u}^k_i , \mathbf{u}^k_i), (\bar{\mathbf{u}}^k_i, \bar{\mathbf{u}}^k_i), (\mathbf{u}^k_i, \bar{\mathbf{u}}^k_i), (\bar{\mathbf{u}}^k_i, \mathbf{u}^k_i)$の形になる。\\
      帰納法の仮定から、 \\
      $0 \leq i \leq 2^k$のとき$\mathbf{u}^k_i, \bar{\mathbf{u}}^k_i$は$\mathbf{e}^k_1, \ldots, \mathbf{e}^k_{k+1}$の線型結合で表せるので、\\
      $1 \leq i \leq k + 1$のもと $\mathbf{e}^{k+1}_i = ( \mathbf{e}^k_i, \mathbf{e}^k_i )$とおけば、\\
      $(\mathbf{u}^k_i , \mathbf{u}^k_i), (\bar{\mathbf{u}}^k_i, \bar{\mathbf{u}}^k_i)$の形の符号は$\mathbf{e}^{k+1}_1, \ldots, \mathbf{e}^{k+1}_{k+1}$の線型結合で表せる。\\
    \end{block}
  \end{frame}

  \begin{frame}{証明 (3/3)}
    \begin{block}{$S^{k+1}$から作られる符号の線形性}
      更に、$C_k$は線形符号なので、$\mathbf{0} = ( 0 \ldots 0 )$を含むため、\\
      $C_{k+1}$は$(\mathbf{0}, \bar{\mathbf{0}}) = (0\ldots01\ldots 1)$を含み、
      これは$(\mathbf{u}^k_i , \mathbf{u}^k_i), (\bar{\mathbf{u}}^k_i, \bar{\mathbf{u}}^k_i)$の形ではない。\\
      $\mathbf{e}^{k+1}_{k+2} = (0\ldots 01 \ldots 1)$とおくと、$1 \leq i \leq 2^k$に対し \\
      $\mathbf{e}^{k+1}_{k+2} + (\bar{\mathbf{u}}^k_i, \bar{\mathbf{u}}^k_i) = (\bar{\mathbf{u}}^k_i, \mathbf{u}^k_i)$かつ$\mathbf{e}^{k+1}_{k+2} + (\mathbf{u}^k_i, \mathbf{u}^k_i) = (\mathbf{u}^k_i, \bar{\mathbf{u}}^k_i)$\\
      よって、$C_k$が線形のとき$C_{k+1}$は$\mathbf{e}^{k+1}_1 , \ldots, \mathbf{e}^{k+1}_{k+2}$を基底として線形となる。
    \end{block}
    
    以上から、数学的帰納法により、Sylvester Matrixから作られるHadamard Codeは線形であることが示された。
  \end{frame}

\end{document}

