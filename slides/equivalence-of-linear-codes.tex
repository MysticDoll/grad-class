\documentclass[dvipdfmx,10pt,jsarticle]{beamer}
\usetheme{CambridgeUS}

\usepackage{amsmath}
\usepackage{amssymb}
\usepackage{newtxtext}
\title{Matrix Description of Linear Codes}
\author{Mitsuru Takigahira}
\date[2017/11/20]{}

\begin{document}
  \frame{\maketitle}
  \begin{frame}{TL;DR}
    線形空間は一般に一意の基底をもつわけではないため、線形符号$\mathcal{C}$生成行列$G$とパリティ検査行列$H$は一般に一意でない。

    $G$及び$H$の形式を可能な限り簡単にすることは実用的であり、例えば$0$の要素が多いほど、計算がより簡単になる。
  \end{frame}

  \begin{frame}{行基本変形による生成行列の変更}
    $G$の行$\mathbf{r}_1, \ldots, \mathbf{r}_k$は$\mathcal{V}$の要素で、$\mathcal{C}$の基底をなすとみなせる。\\
    行の基本変形、つまり以下の操作
    \begin{itemize}
      \item 行の交換
      \item 非零な定数と行との掛け算
      \item 行$\mathbf{r}_i$を$\mathbf{r}_i + a\mathbf{r}_j$で置き換える ($j \neq i, a \neq 0$)
    \end{itemize}
    を$G$に行う場合、$\mathcal{C}$の基底は変更されるが、$G$の各行が張る部分空間は変わらず\\ 
    $\mathcal{C}$のままである。\\
    よって、これらの操作をどんな順番で$G$に対して行っても、 \\
    得られる新たな生成行列も同じ$\mathcal{C}$の生成行列となる。
  \end{frame}

  \begin{frame}{列の入れ替えによる生成行列の変更}
    $G$に対する列の入れ替えは$\mathcal{C}$を変更してしまうが、 \\
    得られた新たな符号と$\mathcal{C}$との違いは、各符号語のシンボルの順序の点のみである。 \\
    よってこの操作で得られた2つの符号は同じ$n, k, d, t, M, R \ \text{etc.}$の値を持ち、 \\
    そのためこれらは基本的にはほとんど変わらない。 \\
    このことは次の定義を導く
    \begin{block}{定義}
    2つの線形符号$\mathcal{C}_1, \mathcal{C}_2$が、それぞれ生成行列$G_1, G_2$を持ち
    \begin{itemize}
      \item 行基本変形
      \item 列の入れ替え
    \end{itemize}
    で一方から他方に変形できるとき、符号$\mathcal{C}_1, \mathcal{C}_2$は同値な符号であるという。 \\
    (列を定数倍したり、列に他の列を定数倍して足す操作は認めない)
    \end{block}

    これは$\mathcal{C}_1$の全符号語のシンボルの順序を一斉に入れ替えることによって \\
    $\mathcal{C}_2$が得られることを意味している。\\
    つまり、$\mathcal{C}_1, \mathcal{C}_2$はそれぞれ異なる符号語から構成されているが、\\ 
    実際には「同じ符号である」と考えてしまうということである。
  \end{frame}

  \begin{frame}{組織符号}
    行の基本変形と列の入れ替えをうまく行えば、どんな生成行列も次の形にできる
    \[ G = \begin{pmatrix}
        I_k & \mid & P
      \end{pmatrix} = \begin{pmatrix}
        1 &  &  &  & \ast & \ast & \cdots & \ast \\
          & 1 &  &  & \ast & \ast & \cdots & \ast \\
          &  & \ddots &  & \vdots & \vdots &  & \vdots \\
          &  &  & 1 & \ast & \ast & \cdots & \ast \\
      \end{pmatrix} \]
      $I_k$は$k \times k$単位行列で、$P$は$\ast$で表された$k$行$n-k$列の行列である。 \\
      このとき、この$G$ (あるいは$\mathcal{C}$)を組織符号形式と呼ぶ。  \\
      この場合、各$\mathbf{a} = a_1 \ldots a_k \in \mathbf{F}_k$は
      \[ \mathbf{u} = \mathbf{a} G = a_1 \ldots a_k a_{k+1} \ldots a_n \]
      に符号化され、$a_1 \ldots a_k$は情報桁、$a_{k+1} \ldots a_n = \mathbf{a} P$は$n - k$桁の検査桁となる。\\
    情報桁は恣意的に決定できるが、一方検査桁は一意に$\mathbf{a}$及び$G$
    によって決定され、$\mathbf{a} P$中のシンボルとして簡単に計算される
  \end{frame}
\end{document}
