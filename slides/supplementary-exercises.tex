\documentclass[dvipdfmx,10pt,jsarticle]{beamer}
\usetheme{CambridgeUS}

\usepackage{amsmath}
\usepackage{amssymb}
\usepackage{newtxtext}
\title{Supplementary Exercises}
\author{Mitsuru Takigahira}
\date[2017/10/20]{}

\begin{document}
  \frame{\maketitle}
 
  \begin{frame}{練習 6.16}
    $\mathbf{Z}_2$上で既約な3次多項式を示し、それを用いて位数8の体$\mathbf{F}_8$を構成せよ。
    そのような多項式がちょうど2つあることを示し、対応する体は同型であることを示せ。
  \end{frame}

  \begin{frame}{練習 6.16: 解答 (1)}
    3次の多項式$f(x)$が既約であることは、 \\一次因子を持たない (すなわち$f(x) = 0$が解をもたない)ことと同値である。 \\
    よって、$\mathbf{F}_2$上で既約な3次多項式は
    \begin{itemize}
      \item $f_\alpha (x) = x^3 + x^2 + 1$
      \item $f_\beta (x) = x^3 + x + 1$
    \end{itemize}
    の2つのみ存在し、解をそれぞれ$\alpha, \beta$とおくと、2つの位数8の有限体
    \begin{itemize}
      \item $\mathbf{F}_\alpha = \lbrace a \alpha^2 + b \alpha + c \mid a, b, c \in \mathbf{F}_2 \rbrace$
      \item $\mathbf{F}_\beta = \lbrace a \beta^2 + b \beta + c \mid a, b, c \in \mathbf{F}_2 \rbrace$
    \end{itemize}
    を構成できる。\\
    このとき$\alpha^3 = \alpha^2 + 1$及び、$\beta^3 = \beta + 1$なので、\\
    ${(\alpha + 1)}^3 = \alpha^3 + \alpha^2 + \alpha + 1 = \alpha = (\alpha + 1) + 1$ となる。これを利用して、
    \[ g: \mathbf{F}_\beta \rightarrow \mathbf{F}_\alpha \]
    \[g(a \beta^2 + b \beta + c) = a {(\alpha + 1)}^2 + b (\alpha + 1) + c = a \alpha^2 + b \alpha + (a + b + c) \]
    なる写像$g$を定義できる。 \\
  \end{frame}

  \begin{frame}{練習 6.16: 解答 (2)}
    この$g$は任意の
    \begin{itemize}
      \item $\gamma \in \mathbf{F}_2$
      \item $x = a_x \beta^2 + b_x \beta + c_x \in \mathbf{F}_\beta$
      \item $y = a_y \beta^2 + b_y \beta + c_y \in \mathbf{F}_\beta$
    \end{itemize}
    に対して
    \begin{itemize}
      \item $ g(x + y) = (a_x + a_y) \alpha^2 + (b_x + b_y) \alpha + (a_x + a_y + b_x + b_y + c_x + c_y) = g(x) + g(y) $
      \item $ g(\gamma x) = \gamma a_x \alpha^2 + \gamma b_x \alpha + \gamma a_x + \gamma b_x + \gamma c_x = \gamma(a_x \alpha^2 + b_x \alpha + a_x + b_x + c_x) = \gamma g(x) $
    \end{itemize}
    が成立するため線形写像である。\\
    また、任意の$x,y \in \mathbf{F}_\beta$に対して、$g(x) = g(y) \Rightarrow g(x) - g(y) = g(x - y) = 0 \therefore x = y$が成立し単射であり、\\
    更に$\mid \mathbf{F}_\beta \mid = \mid \mathbf{F}_\alpha \mid = 8$から全単射であることが言えるので$g$は同型写像である。 \\
    以上より、2つの位数8の有限体$\mathbf{F}_\alpha, \mathbf{F}_\beta$が同型であることが示された。

  \end{frame}

  \begin{frame}{練習 6.17}
    $p \equiv 3 \mod (4)$なる各素数$p$に対して、多項式$f(x) = x^2 + 1$が\\ $\mathbf{Z}_p$上で既約であることを示せ。\\
    これを用いて$q = p^2$なる位数$q$の体$\mathbf{F}_q$を構成せよ。\\
    どんな素数$p$に対して多項式$x^2 + x + 1$から位数$q = p^2$の体を構成できるか?
  \end{frame}

  \begin{frame}{練習 6.17: 解答}
    $x^2 + 1$が可約と仮定する、このとき$x^2+ 1 = 0$はなんらかの解$\alpha \in \mathbf{Z}_p$を持つ。\\
    式から$\alpha^2 = -1 $なので$\alpha^4 = 1$となり、フェルマーの小定理から$4 \mid p-1$である。\\
    これは$p \equiv 3 \mod (4)$に矛盾する。よって多項式$x^2 + 1$は$\mathbf{Z}_p$上で既約である。 \\
    よって$x^2 + 1$の解$\alpha$を$\mathbf{Z}_p$に添加し、体$\mathbf{F}_{p^2} = \lbrace a \alpha + b \mid a, b \in \mathbf{Z}_p \rbrace$を構成できる。 \\
    同様の議論から、ある素数$p$に対して$\mathbf{Z}_p$上で$x^2 + x + 1$が可約ということは、$\alpha^3 - 1 = (\alpha - 1)(\alpha^2 + \alpha + 1) = 0$なる$\mathbf{Z}_p \ni \alpha \neq 1$が存在し、\\
    フェルマーの小定理から$3 \mid p - 1$となる。\\
    よって$p \equiv 2 \mod (3)$なる$p$に対して与式は既約となり、その解$\alpha$を用いて$\mathbf{F}_{p^2} = \lbrace a \alpha + b \mid a, b \in \mathbf{Z}_p \rbrace$を構成できる。
  \end{frame}

  \begin{frame}{練習 6.18}
    シングルトン限界、つまり\\
    $\mathbf{F}_q$上の符号長$n$かつ最小距離$d$で符号語数が$M$の符号$C$に対して、$\log M \leq n - d + 1$が成り立つことを証明せよ。\\
    この上界に達する符号を最大分離符号 (MDS)と呼ぶが、このような符号はどのようなものか?
    \begin{block}{解答}
      $C \subseteq \mathbf{F}_q^n$より$M \leq q^n$で、最小距離が$d$なので、少なくとも$d - 1$回符号桁を削除してパンクチャド符号を構成できる。\\
      これにより、$M \leq q^{n - d + 1}$が成立し、$\log M \leq n - d + 1$も成り立つ。 \\
      反復符号$\mathcal{R}_n$及びパリティ検査符号$\mathcal{P}_n$がこの上界に達する。
    \end{block}
  \end{frame}

  \begin{frame}{練習 6.19}
    次の計算と因数分解を行い、その結果から完全符号の存在についてどのようなことが言えるか答えよ
    \[ 1 + \binom{23}{1} + \binom{23}{2} + \binom{23}{3} \  \text{and} \  1 + \binom{11}{1}\cdot 2 + \binom{11}{2} \cdot 2\]
    \begin{block}{解答}
      \[1 + \binom{23}{1} + \binom{23}{2} + \binom{23}{3} = 2048 = 2^{^{11}} \ , \ 1 + \binom{11}{1}\cdot 2 + \binom{11}{2} \cdot 2 = 243 =  3^5\]
      より、前者は$q = 2, M = 2^{12}, t = 3$の、後者は$q = 3, M = 3^6, t = 2$の\\ハミングの球充填限界式の等号である。\\
      よって符号語長$23$の2元完全符号及び符号語長$11$の3元完全符号の存在が示された
    \end{block}
  \end{frame}

  \begin{frame}{練習 6.20}
    $i = 1, 2$に対して$C_i$が$\mathcal{V} = \mathbf{F}_q^n$上の$(n, M_i, d_i)$符号のとき、
    \[ C_1 \oplus C_2 = \lbrace(\mathbf{x} , \mathbf{y}) \in \mathcal{V} \oplus \mathcal{V} \mid \mathbf{x} \in C_1, \mathbf{y} \in C_2 \rbrace \]
    が$d = \min (d_1, d_2)$の下$(2n, M_1M_2, d)$符号であることを示せ。更に
    \[ C_1 \ast C_2 = \lbrace(\mathbf{x} , \mathbf{x} + \mathbf{y}) \in \mathcal{V} \oplus \mathcal{V} \mid \mathbf{x} \in C_1, \mathbf{y} \in C_2 \rbrace \]
    が$d = \min (2d_1, d_2)$の下$(2n, M_1M_2, d^\prime)$符号であることを示せ。\\
    また、各$C_i$が次元$k_i$で線形の場合$C_1 \oplus C_2$と$C_1 \ast C_2$はともに線形で次元が$k_1 + k_2$となることを示せ。
  \end{frame}
  \begin{frame}{練習 6.20: 解答 (1)}
    $C_1 \oplus C_2$の元を最も近づけるには、
    \begin{itemize}
      \item $\mathbf{x} \in C_1$と$d(\mathbf{y}_1, \mathbf{y}_2) = d_2$なる$\mathbf{y}_1, \mathbf{y}_2 \in C_2$からなる符号語$(\mathbf{x}, \mathbf{y}_1), (\mathbf{x}, \mathbf{y}_2)$
      \item $\mathbf{y} \in C_2$と$d(\mathbf{x}_1, \mathbf{x}_2) = d_1$なる$\mathbf{x}_1, \mathbf{x}_2 \in C_2$からなる符号語$(\mathbf{x}_1, \mathbf{y}), (\mathbf{x}_2, \mathbf{y})$
    \end{itemize}
    のどちらかとなるため、符号$C_1 \oplus C_2$の最小距離は$d = \min(d_1, d_2)$ \\
    同様に$C_1 \ast C_2$も同じ$\mathbf{x}, \mathbf{x}_1, \mathbf{x}_2, \mathbf{y}, \mathbf{y}_1, \mathbf{y}_2$を用いて、\\
    \begin{itemize}
      \item $(\mathbf{x}_1, \mathbf{x}_1 + \mathbf{y}), (\mathbf{x}_2, \mathbf{x}_2 + \mathbf{y})$
      \item $(\mathbf{x}, \mathbf{x} + \mathbf{y}_1), (\mathbf{x}, \mathbf{x} + \mathbf{y}_2)$
    \end{itemize}
    のどちらかが最小距離だけ離れた符号語の組となる。このとき、\\
    それぞれの組の距離は
    \begin{itemize}
      \item $d((\mathbf{x}_1, \mathbf{x}_1 + \mathbf{y}), (\mathbf{x},_2 \mathbf{x}_2 + \mathbf{y})) = \text{w}_H((\mathbf{x}_1 - \mathbf{x}_2, \mathbf{x}_1 - \mathbf{x}_2)) = 2d_1$
      \item $d((\mathbf{x}, \mathbf{x} + \mathbf{y}_1), (\mathbf{x}, \mathbf{x} + \mathbf{y}_2)) = \text{w}_H((\mathbf{0}, \mathbf{y}_1 - \mathbf{y}_2)) = d_2$
    \end{itemize}
    となり、$C_1 \ast C_2$の最小距離は$\min(2d_1, d_2)$となる。
  \end{frame}

  \begin{frame}{練習 6.20: 解答 (2)}
    また、各$C_i$が次元$k_i$で線形の場合、$\mathbf{0} \in C_i \ \text{for} \ {}^\forall i \in \lbrace 1, 2 \rbrace$で、\\
    各$C_i$の基底を$\mathbf{e}_j^{(i)} \ (j = 1 , \ldots, k_i)$とおくと、$C_1 \oplus C_2$の任意の元は \\
    $(\mathbf{e}_1^{(1)},\mathbf{0}) , \ldots, (\mathbf{e}_{k_1}^{(1)}, \mathbf{0}), (\mathbf{0}, \mathbf{e}_1^{(2)}), \ldots, (\mathbf{0}, \mathbf{e}_{k_2}^{(2)})$の線型結合で表せるため線形で、 \\
    次元は$k_1 + k_2$となる。\\
    同様にして$C_1 \ast C_2$の任意の元も$(\mathbf{e}_1^1,\mathbf{e}_1^1) , \ldots, (\mathbf{e}_{k_1}^1, \mathbf{e}_{k_1}^1), (\mathbf{0}, \mathbf{e}_1^2), \ldots, (\mathbf{0}, \mathbf{e}_{k_2}^2)$ \\
    の線型結合で表せるため線形で次元は$k_1 + k_2$となる。
    
  \end{frame}

  \begin{frame}{練習 6.21}
    ハイフンを無視する場合、国際標準図書番号 (ISBN)は$\mathbf{Z}_{11} = \lbrace 0, 1, \ldots, 9, X\rbrace$ ($X$は$10$を表す)上の長さ$10$の符号語$w = a_1 \ldots a_{10}$である。\\
    $a_1, \ldots, a_9$の桁は情報桁で、国や出版社などの情報を示している。 \\ $a_{10}$は$a_1 + 2a_2 + \cdots + 10 a_{10} \equiv 0 \mod (11)$で定義される検査桁である。 \\
    この符号がどんな1つの誤りと、更に2桁の入れ替えを検出することを示せ (これらは一般的な人的エラーである)。 次のどれが正しいISBNか? \\
    \begin{center}
        3-540-76197, \ 3-540-76179-7, \ 3-541-76197-7 
    \end{center}
  \end{frame}
\end{document}
