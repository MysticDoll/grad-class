\documentclass[dvipdfmx,10pt,jsarticle]{beamer}
\usetheme{CambridgeUS}

\usepackage{tikz}
\usepackage{amsmath}
\usepackage{amssymb}
\usepackage{newtxtext}
\title{Hadamard Matrices}
\author{Mitsuru Takigahira}
\date[2017/09/29]{}

\begin{document}
  \frame{\maketitle}
  \begin{frame}{アダマール行列と符号}
    多くの数学的構造は符号を作るために利用できる。ある興味深い種類の符号はアダマール行列と呼ばれる行列から作れる。
    最初にこれらの行列の初歩的な性質について調べよう (詳細は[Ha67, MS77]を見よ) \\

  \end{frame}

  \begin{frame}{定義 アダマール行列}
    アダマールは与えられた各$n$に対し、\\ $n \times n$実行列$H$の行列式がどれだけ大きく出来るかに興味を持った。\\
    この問題に意味づけるために$H$の要素に制限が必要であるが、 \\
    全ての$i, j$に対して、$\mid h_{ij} \mid \leq 1$としても一般性を失わない。\\
    これらの条件の下、アダマールは$\mid \det H \mid \leq n^{n/2}$の等号成立が
    \begin{itemize}
      \item[(a)] 各$h_{ij} = \pm 1$ かつ
      \item[(b)] $H$の相異なる行$\mathbf{r}_i$は直交する、つまり$i \neq j$なる全ての$i, j$に対し、$\mathbf{r}_i \ldotp \mathbf{r}_j = 0$
    \end{itemize}
    の必要十分条件であることを証明した。 \\
  \end{frame}

  \begin{frame}{定義 アダマール行列}
    (a) 及び (b) を満たす$n \times n$行列$H$は、$n$次アダマール行列と呼ばれる。\\
    (a)は全ての$i$に対して$\mathbf{r}_i \ldotp \mathbf{r}_i = n$を意味し、$H H^T$が対角行列
    \[ H H^T = \begin{pmatrix}
        n & 0 & \cdots & 0 \\
        0 & n & \cdots & 0 \\
        \vdots & \vdots & \ddots & \vdots \\
        0 & 0 & \cdots & n
    \end{pmatrix} = n I_n; \eqno{ (6.8)}\]
    であることがわかる。ここで、\\ $H^T$は$H$の転置行列を意味し、$I_n$は$n \times n$の単位行列である。
  \end{frame}
  \begin{frame}{定義 アダマール行列}
    $\det H^T = \det H$より、 (6.8)から
    \[ {(\det H)}^2 = \det (nI_n) = n^n \]
    よって、$\mid \det H \mid = n^{n/2}$である。\\
    このことから、全てのアダマール行列はアダマールの上界に達する。\\
    この逆の証明は難しく、ここでは必要ないので省略する。
  \end{frame}

  \begin{frame}{例 6.23}
    見やすさと印刷上の理由により、以下ではアダマール行列の$-1$の成分を単に$-$と記述する。

    \begin{block}{例 6.23}
      行列$H = \begin{pmatrix} 1 \end{pmatrix}$と$\begin{pmatrix} 1 & 1 \\ 1 & -\end{pmatrix}$はそれぞれ$1$次と$2$次のアダマール行列で、$\mid \det H \mid = 1$及び$2$である。
    \end{block}
    \vspace{1cm}
    \begin{block}{練習 6.12}
        全ての$1$次及び$2$次のアダマール行列を求めよ。 \\
    \end{block}
  \end{frame}

  \begin{frame}{補題 6.24}
    次の簡単な結果によって、大きなアダマール行列を小さなアダマール行列から作ることが出来る。

    \begin{block}{補題 6.24}
      $H$を$n$次のアダマール行列とおく、そして$H^\prime$を
      \[H^\prime = \begin{pmatrix} H & H \\ H & -H\end{pmatrix}\]
      と置く。このとき、$H^\prime$は$2n$次のアダマール行列となる。
    \end{block}
    \vspace{1cm}
     \begin{block}{練習 6.13}
        補題 6.24を証明せよ。
    \end{block}
  \end{frame}

  \begin{frame}{系 6.25}
    \begin{block}{系 6.25}
      各整数$m \geq 0$に対して$2^m$次のアダマール行列が存在する。
    \end{block}
    \vspace{1cm}
      
    \begin{block}{証明}
        $H = \begin{pmatrix}  1\end{pmatrix}$から始め、補題 6.24を$m$回適用すれば良い。
    \end{block}
  \end{frame}

  \begin{frame}{例 6.26}
    補題 6.24の方法で得られる $2^m$次のアダマール行列は \\シルベスター行列 (Sylvester matrices)と呼ばれている。\\
    例えば$m = 1$を取ると、$\begin{pmatrix} 1 & 1 \\ 1 & - \end{pmatrix}$を与え、そして$m = 2$に対して以下を得る。
      \[ \begin{pmatrix}
          1 & 1 & 1 & 1 \\
          1 & - & 1 & - \\
          1 & 1 & - & - \\
          1 & - & - & 1
      \end{pmatrix} \]

    しかしながら、アダマール行列は全ての次数で存在するわけではない。例えば$n > 1$なる奇数次のアダマール行列は存在しない。
  \end{frame}

  \begin{frame}{補題 6.27}
    \begin{block}{補題 6.27}
      $n > 1$なる$n$次のアダマール行列$H$が存在する場合、$n$は偶数である。
    \end{block}

      \begin{block}{証明}
        $i \neq j$に対して行$\mathbf{r}_i$と$\mathbf{r}_j$の直交性は$h_{i1}h_{j1} + \cdots + h_{in}h_{jn} = 0$を与える。 \\
        それぞれの$h_{ik}h_{jk} = \pm1$で、よって$n$は偶数でなければならない。
      \end{block}
  \end{frame}

  \begin{frame}{補題 6.28}
    \begin{block}{補題 6.28}
      $n > 2$なる$n$次のアダマール行列$H$が存在する場合、$n$は$4$で割り切れる。
    \end{block}

      \begin{block}{証明}
        $H$の任意の異なる行$\mathbf{r}_i, \mathbf{r}_j$に対して$k$番目の要素に$-1$をかけたものをそれぞれ$\mathbf{r}_i^\prime, \mathbf{r}_j^\prime$としたとき、
        \[ \mathbf{r}_i \bullet \mathbf{r}_j^T = r_{i1}r_{j1} + \cdots + r_{ik}r_{jk} + \cdots r_{in}r_{jn}
        = r_{i1}r_{j1} + \cdots + (-r_{ik})(-r_{jk}) + \cdots + r_{in}r_{jn} = \mathbf{r}_i^\prime \bullet \mathbf{r}_j^\prime \]
        となるので、任意の行に$-1$をかけた行列はアダマール行列の性質を失わず、 \\そのため最初の行$\mathbf{r}_1$は全ての要素が$1$と仮定して良い。\\
        各行$\mathbf{r}_i \ (i \neq 1)$は$\mathbf{r}_1$と直交し、$n/2$個の要素は$1$で、残りの$n/2$個は$-1$である。 \\
        列を交換すると (これも同様の議論からアダマール行列の性質を失わない) 、 \\ 以下のように仮定できる
        \[ \mathbf{r}_2 = \begin{pmatrix} 1 & 1 & \ldots & 1 & -1 & -1 & \ldots & -1 \end{pmatrix} \]
      \end{block}
  \end{frame}
  \begin{frame}{証明: 補題 6.28}
      \begin{block}{証明}
        次に$\mathbf{r}_1, \mathbf{r}_2$のそれぞれと異なる$3$つ目の行$\mathbf{r}_3$について考える \\
        $\mathbf{r}_3$の列の最初と最後の$n/2$個を要素がそれぞれ$1$を$u$個と$v$個含むとする (そして残りの要素が$-1$となる)。 このとき
        \[0 = \mathbf{r}_1 \ldotp \mathbf{r}_3 = u - \big( \frac n2 - u \big) + v - \big( \frac n2 - v \big) = 2u + 2v - n\]
        さらに、
        \[ 0 = \mathbf{r}_2 \ldotp \mathbf{r}_3 = u - \big( \frac n2 - u \big) - v + \big( \frac n2 - v \big) = 2u - 2v \]
        よって、$u = v$で、それゆえに$n = 2u + 2v = 4u$は$4$で割り切れる。
      \end{block}
      この逆、つまり$4$で割り切れる$n$に対して、$n$次のアダマール行列が存在することが推測できる。
      これは未だに未解決問題である。
  \end{frame}

  \begin{frame}{定理 6.29}
    符号理論とアダマール行列の関係性は次の結果に基づいている。
    \begin{block}{定理 6.29}
      それぞれ$n$次のアダマール行列$H$から符号長$n$で符号語数$M = 2n$、最小距離$n/2$の二元符号を構成できる。
    \end{block}

  \end{frame}

  \begin{frame}{証明: 定理 6.29}
      \begin{block}{アダマール行列からの符号の構成}
        \begin{enumerate}
          \item $2n$個のベクトル$\pm\mathbf{r}_1 , \ldots , \pm\mathbf{r}_n \in \mathbf{R}^n$を$H$の各行$\mathbf{r}_i$から構成する
            \begin{itemize}
              \item 行の直交性からこれらのベクトルは全て互いに素である。
            \end{itemize}
          \item $-1$の要素を$0$に置き換えることにより、$0, 1$の要素からなる$2n$個のベクトルを得る。
            \begin{itemize}
              \item これらのベクトルは$\mathcal{V} = \mathbf{F}_2^n$の元とみなすことが出来るので、これらは2元符号$C$となる。
            \end{itemize}
        \end{enumerate}
      \end{block}
        以上の構成法によって構成された任意の符号$C$は符号長$n$のアダマール符号と呼ばれ、
        このような符号で符号長$32$のものは1969年火星探査機マリナーからの写真伝送に使われた。 \\
  \end{frame}
  \begin{frame}{アダマール符号の性質}
        アダマール符号は以下のような性質を持つ。
      \begin{block}{アダマール符号の性質}
        \begin{itemize}
          \item 符号語は$\bar{\mathbf{u}_i} = \mathbf{1} - \mathbf{u}_i$ のもと $\mathbf{u}_1, \bar{\mathbf{u}_1}, \ldots , \mathbf{u}_n, \bar{\mathbf{u}_n}$の形になる。
          \item $C$は最小距離$d = n / 2$を持つ。
            \begin{itemize}
              \item 任意の$i$に対して、$d(\mathbf{u}_i, \bar{\mathbf{u}_i}) = n$
                \begin{itemize}
                  \item $\mathbf{u}_i$と$\bar{\mathbf{u}_i}$は全ての要素が異なっているため
                \end{itemize}
              \item $i \neq j \Rightarrow d(\mathbf{u}_i, \mathbf{u}_j) = d(\bar{\mathbf{u}_i}, \mathbf{u}_j) = d(\mathbf{u}_i, \bar{\mathbf{u}_j}) = d(\bar{\mathbf{u}_i}, \bar{\mathbf{u}_j}) = n/2$
                \begin{itemize}
                  \item アダマール行列の性質 (b)より各行の要素の$n/2$が$1$、残りの$n/2$が$0$となるため
                \end{itemize}
            \end{itemize}
        \end{itemize}
      \end{block}
  \end{frame}

  \begin{frame}{練習問題}
        \begin{block}{練習 6.14}
          例 6.26のアダマール行列$H$から上記の方法によって得られる全ての符号語を求めよ。これらは線形符号か? \\
        \end{block}


        \begin{block}{練習 6.15}
          $8$次のアダマール行列を構成し、符号長$8$のアダマール符号を構成せよ。\\
          この符号の伝送速度はどうなるか?この符号はどれだけの誤りを訂正できるか?そしてどれだけの誤りを検出するか? \\
        \end{block}
  \end{frame}

  \begin{frame}{アダマール符号の性質}
    $n$が$2$の類乗数でない場合、$2n$もまた$2$の類乗数ではない。 \\
    よってそのような$n$に対して符号長$n$のアダマール符号は線形にはなりえない。\\
    任意の符号長$n$のアダマール符号の伝送速度は
    \[ R = \frac{\log_2 (2n)}{n} = \frac{1 + \log_2 n}{n} \rightarrow 0 \ \text{as} \ n \rightarrow \infty \]
    訂正可能な誤りの数は ($n > 2$ の場合) 定理6.10 と 定理6.29 と 系6.28から
    \[t = \lfloor \frac{d-1}{2} \rfloor = \lfloor \frac{n -2}{4} \rfloor = \frac n4 - 1 \]
    なので、よって訂正される誤りの割合は
    \[ \frac tn = \frac 14 - \frac 1n \rightarrow \frac 14 \ \text{as} \ n \rightarrow \infty \]
    となる。
  \end{frame}


\end{document}
