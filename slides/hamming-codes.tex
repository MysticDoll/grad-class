\documentclass[dvipdfmx,10pt,jsarticle]{beamer}
\usetheme{CambridgeUS}

\usepackage{amsmath}
\usepackage{amssymb}
\usepackage{newtxtext}
\title{The Hamming Codes}
\author{Mitsuru Takigahira}
\date[2017/12/01]{}
\newcommand{\F}{\mathbf{F}}
\newcommand{\code}[1]{\mathcal{#1}}
\newcommand{\vs}[1]{\mathcal{#1}}
\newcommand{\sets}[1]{\lbrace{}  #1 \rbrace}
\newcommand{\bracket}[1]{\lbrack{} #1 \rbrack}
\newcommand{\vcode}[2]{$\bracket{#1 , #2}$}
\renewcommand{\vec}[1]{\mathbf{#1}}

\begin{document}
  \frame{\maketitle}
  \begin{frame}{TL;DR}
    ハミング\vcode74符号$\code{H}_7$は1重誤り訂正完全2元符号で、伝送速度$\frac 47$である。 \\
    実際これは1重誤り訂正2元符号の無限列のうちの1つであり、\\
    この列は符号長$n$が増えるほど伝送速度$R$は1に近づく。 \\
    この符号はハミングによって1950年に考案された\footnote{[Ha50]}が、 \\
    ゴレイも同時期に独自にこれらを発見した \\
    (どちらが先に発見されたかの議論については[Th83]を見よ。)
  \end{frame}

  \begin{frame}{ハミング符号の構成}
    1重誤り訂正2元符号に対して、ハミングの球充填限界式 (系 6.17)において\\
    $t = 1$及び$q = 2$とおくため、 完全符号に対して以下の条件が成立する。
    \[ 2^{n-k} = 1 + \binom n 1 = n + 1 \]
    $c = n-k$ (検査桁の桁数)とおけば、この条件は以下の式と同値である
    \[ n = 2^c - 1 \ \eqno{(7.4)} \]
    $k = n - c = 2^c - c$なので、$n$及び$k$のとりうる値は次のようになる。
    \begin{align*}
      \begin{array}{cccccccc}
        c & = & 1 & 2 & 3 & 4 & 5 & \cdots \\
        n & = & 1 & 3 & 7 & 15 & 31 & \cdots \\
        k & = & 0 & 1 & 4 & 11 & 26 & \cdots 
      \end{array}
    \end{align*}
  \end{frame}
  \begin{frame}{ハミング符号の構成}
    このようなパラメータに対して符号を構成しよう。 \\
    $t = 1$とおき、系 7.31からそのような符号$\code{C}$が存在することは \\
    \begin{itemize}
      \item 階数$c$
      \item すべての2つの列の組が線形独立となっている
    \end{itemize}
    なる$c \times n$行列$H$が$\F_2$上に存在することと同値で、$\F = \F_2 = \sets{0, 1}$より、
    \begin{itemize}
      \item $H$の任意の列$\vec{c}_i$が非零
      \item $H$のすべての列が互いに異なっている
    \end{itemize}
    つまり$H$は互いに異なる$n = 2^c - 1$個の非零な長さ$c$の列ベクトルからなる。 \\
    このとき、$2^c$個の相異なる長さ$c$の2元ベクトルのみが存在し、列$\vec{c}_i$の選択は存在しない。
    これらは何らかの順の非零な長さ$c$の列すべて ($2^c-1$個)になる。 \\
    ($c = 3$の場合については例 7.3を見よ)
  \end{frame}

\end{document}
