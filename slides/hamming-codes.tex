\documentclass[dvipdfmx,10pt,jsarticle]{beamer}
\usetheme{CambridgeUS}

\usepackage{amsmath}
\usepackage{amssymb}
\usepackage{newtxtext}
\title{The Hamming Codes}
\author{Mitsuru Takigahira}
\date[2017/12/01]{}
\newcommand{\F}{\mathbf{F}}
\newcommand{\code}[1]{\mathcal{#1}}
\newcommand{\vs}[1]{\mathcal{#1}}
\newcommand{\sets}[1]{\lbrace{}  #1 \rbrace}
\newcommand{\bracket}[1]{\lbrack{} #1 \rbrack}
\newcommand{\vcode}[2]{$\bracket{#1 , #2}$}
\renewcommand{\vec}[1]{\mathbf{#1}}

\begin{document}
  \frame{\maketitle}
  \begin{frame}{TL;DR}
    ハミング\vcode74符号$\code{H}_7$は1重誤り訂正完全2元符号で、伝送速度$\frac 47$である。 \\
    実際これは1重誤り訂正2元符号の無限列のうちの1つであり、\\
    この列は符号長$n$が増えるほど伝送速度$R$は1に近づく。 \\
    この符号はハミングによって1950年に考案された\footnote{[Ha50]}が、 \\
    ゴレイも同時期に独自にこれらを発見した \\
    (どちらが先に発見されたかの議論については[Th83]を見よ。)
  \end{frame}

  \begin{frame}{ハミング符号の構成}
    1重誤り訂正2元符号に対して、ハミングの球充填限界式 (系 6.17)において\\
    $t = 1$及び$q = 2$とおくため、 完全符号に対して以下の条件が成立する。
    \[ 2^{n-k} = 1 + \binom n 1 = n + 1 \]
    $c = n-k$ (検査桁の桁数)とおけば、この条件は以下の式と同値である
    \[ n = 2^c - 1 \ \eqno{(7.4)} \]
    $k = n - c = 2^c - c$なので、$n$及び$k$のとりうる値は次のようになる。
    \[
      \begin{array}{cccccccc}
        c & = & 1 & 2 & 3 & 4 & 5 & \cdots \\
        n & = & 1 & 3 & 7 & 15 & 31 & \cdots \\
        k & = & 0 & 1 & 4 & 11 & 26 & \cdots 
      \end{array}
    \]
  \end{frame}
  \begin{frame}{ハミング符号の構成}
    このようなパラメータに対して符号を構成しよう。 \\
    $t = 1$とおき、系 7.31からそのような符号$\code{C}$が存在することは \\
    \begin{itemize}
      \item 階数$c$
      \item すべての2つの列の組が線形独立となっている
    \end{itemize}
    なる$c \times n$行列$H$が$\F_2$上に存在することと同値で、$\F = \F_2 = \sets{0, 1}$より、
    \begin{itemize}
      \item $H$の任意の列$\vec{c}_i$が非零
      \item $H$のすべての列が互いに異なっている
    \end{itemize}
    つまり$H$は互いに異なる$n = 2^c - 1$個の非零な長さ$c$の列ベクトルからなる。 \\
    このとき、$2^c$個の相異なる長さ$c$の2元ベクトルのみが存在し、列$\vec{c}_i$の選択は存在しない。
    これらは何らかの順の非零な長さ$c$の列すべて ($2^c-1$個)になる。 \\
    ($c = 3$の場合については例 7.3を見よ) \\
    これらの列ベクトルは$c$個の標準基底ベクトルを含み、それらは線形独立である。 \\
    よってこの行列$H$は階数$c$を持つ。ここから$\code{C}$の存在が示され、\\
    これらのパラメータを持つどんな2つの線形符号も (列の交換の下)同値となる。 \\
    $\code{C}$を符号長$n = 2^c - 1$の2元ハミング符号$\code{H}_n$と呼ぶ。 \\
    (厳密にはここでは符号長$n$の符号の集合を構成したが、それらはすべて同値で、単一の符号$\code{H}_n$とみなされる傾向がある。)
  \end{frame}

  \begin{frame}{例}
    \begin{block}{例 7.32}
      $\code{H}_1$は符号長1の単一の符号語$0$からなるつまらないケースなので、\\
      $c=1$については無視する。 $c=2$のとき、$n=3$で、
      \[ H = \begin{pmatrix}
          0 & 1 & 1 \\
          1 & 0 & 1
      \end{pmatrix}; \]
      なので、$\code{H}_3$は$\vec{v} v_1 v_2 v_3$に対して、$v_1 + v_3 = v_2 + v_3 = 0$つまり$v_1 = v_2 = v_3$なる \\
      2元の符号語からなる。よってこれは2元反復符号$\code{R}_3$である。\\
      $c = 3$の場合、既に考えたハミング符号$\code{H}_7$を得る。 \\
      $c \geq 4$に対して、符号長$n = 2^c - 1$の新たな完全符号$\code{H}_n$を無限に得る。 \\
      これらの符号は$c \rightarrow \infty$伝送速度
      \[ R = \frac kn = \frac{2^c - 1 - c}{2^c - 1} \rightarrow 1 \]
      となるが、それらはただ1つの誤りしか訂正しない。よって$\text{Pr}_E \not\rightarrow 0$である\\
      (練習 7.4を見よ)
    \end{block}
  \end{frame}

  \begin{frame}{練習}
    \begin{block}{練習 7.4}
      2元ハミング符号$\code{H}_7$に対して、$\Gamma$を$P > \frac 12$のBSC、$\Delta$を最近傍復号とおくとき、 \\
      $\text{Pr}_E$を示せ。$n \rightarrow \infty$のとき$\text{Pr}_E$はどうなるか?
    \end{block}
    \begin{block}{解答}
      \begin{align*}
        \text{Pr}_E &= 1 - \sum_{i=0}^{1} \binom ni P^{n-i} {(1-P)}^i  \\
        &= 1 - (P^{n} {(1-P)}^0 + n P^{n-1} {(1-P)}^1) \\
        &= 1 - P^n - nP^{n-1}(1 - P)
      \end{align*}
      となり、$P < 1$のとき、$\text{Pr}_E \rightarrow 1 \ (n \rightarrow \infty)$、$P = 1$のとき、$\text{Pr}_E \rightarrow 0 \ (n \rightarrow \infty) $
    \end{block}
  \end{frame}
  \begin{frame}{ハミング符号の最近傍復号}
    ハミング符号の最近某複合はとても簡単である。 \\
    \vspace{1em}
    $\code{H}_n$は$t=1$の完全符号で、すべての重み1までの誤りパターン$\vec{e}$を訂正する。\\
    $\vec{u} \in \code{H}_n$が伝送され、$\text{wt}(\vec{e})$として$\vec{v} = \vec{u} + \vec{e}$が受信されたと仮定する。 \\
    よって$\vec{e} =\vec{0}$あるいは$\vec{e}$は$\vec{e}_i \in \vs{V}$の標準基底ベクトル$\vec{e}_i$となる。 \\
    \vspace{1em}
    受信者は$\vec{s} = \vec{v} H^T$を計算し、これを$\vec{v}$のシンドローム\footnote{{\scriptsize 医学用語で、シンドロームとは患者のどこが悪いかの兆候のこと}}
    と呼ぶ。今、$\vec{v}H^T = (\vec{u} + \vec{e}) H^T = \vec{u}H^T + \vec{e}H^T = \vec{e}H^T$ ($\vec{u}H^T = \vec{0}$なので)で、\\
    かつこれは$\vec{e} = \vec{0}$及び$\vec{e} = \vec{e}_i$に対してそれぞれ$\vec{0}$または$\vec{c}^T_i$となる。\\
    \vspace{1em}
    もし$\vec{s} = \vec{0}$ならば、受信者は$\vec{v}$を$\Delta (\vec{v}) = \vec{v} = \vec{u}$として復号する \\
    $\vec{s}= \vec{c}^T_i$ならば、$\Delta(\vec{v}) = \vec{v} - \vec{e}_i$となり$\vec{v}$の$i$番目を変更することで得られる。 \\
    この方法は$\text{wt}(\vec{e}) \leq 1$のときは常に正しく復号するが、\\
    $\text{wt}(\vec{e}) >1$の場合一切正しく復号しない。\\
    何らかの$\vec{u}^\prime \in \code{H}_n \ (\vec{u}^\prime \neq \vec{u})$に対し、$\text{wt}(\vec{e}) \leq 1$のもと$\vec{v} = \vec{u}^\prime + \vec{e}^\prime$のとき、 \\
    上記のアルゴリズムは$\vec{v}$を$\Delta (\vec{v}) = \vec{u}^\prime$として誤って復号してしまう。

  \end{frame}

  \begin{frame}{例}
    \begin{block}{例 7.33}
      以下の組織符号形式のパリティ検査行列を持つ$\code{H}_7$を用いる。
      \[ \begin{pmatrix}
          0 & 1 & 1 & 1 & 1 & 0 & 0 \\
          1 & 0 & 1 & 1 & 0 & 1 & 0 \\
          1 & 1 & 0 & 1 & 0 & 0 & 1
      \end{pmatrix} \]
      $\vec{u} = 1101001 \in \code{H}_7$が送信され、$\vec{v} = 1101101 \in \vs{V}$が受信されたとする。 \\
      つまり誤りパターンは$\vec{e} = \vec{e}_5$であり、シンドロームは$\vec{s} = \vec{v} H^T = 100$で、\\
      $H$の5番目の列の転置$\vec{c}_5^T$である。これは5番目に誤りがあることを示していて、$\Delta (\vec{v}) = 1101001 = \vec{u}$を得る。 \\
      一方で、$\vec{v}^\prime= 1001101$が受信された場合、誤りパターンは$\vec{e}^\prime = \vec{e}_2 + \vec{e}_5$で、\\
      シンドロームは$\vec{s}^\prime = 001 = \vec{c}_7^T$となり、誤りの位置が7番目にあることを示し、\\
      $\Delta(\vec{v}) = 1001100 \ne \vec{u}$となる。\\
      このように、2つの誤りを訂正する代わりに符号は3番目の誤りを作ってしまう。
    \end{block}
  \end{frame}

  \begin{frame}{練習}
    \begin{block}{練習 7.5}
      例 7.33のパリティ検査行列を用いて$\vec{u} = 1100110$が$\code{H}_7$の符号語かを検査せよ。 \\
      この符号語$\vec{u}$が送信され、$\vec{v}=1000110$が受信されたと仮定し、\\
      シンドロームと$\Delta (\vec{v})$を求めよ。 \\
      $\vec{v}^\prime = 0000110$が受信された場合何が起こるかを調べ、説明せよ。
    \end{block}
    \begin{block}{解答}
      板書する。
    \end{block}
  \end{frame}

  \begin{frame}{ハミング符号の復号}
    $H$の各列$\vec{c}_i$の順番$\vec{c}_1^T, \ldots \vec{c}_n^T$が$i = 1,\ldots,n$の2進数表現となっている場合、\\
    $\code{H}_7$の復号は特に簡単で、$n = 7$の場合$H$は例 7.13にあるような以下の形になる
    \[ \begin{pmatrix}
        0 & 0 & 0 & 1 & 1 & 1 & 1 \\
        0 & 1 & 1 & 0 & 0 & 1 & 1 \\
        1 & 0 & 1 & 0 & 1 & 0 & 1
    \end{pmatrix} \]
    これは置換$(1362547)$を例 7.33で用いられたパリティ検査行列の列に対して\\
    適用したものと同値である。\\
    シンドローム$\vec{s} = \vec{0}$のとき$\vec{e} = \vec{0}$と解釈され、つまり誤りは無く、 \\
    単一の誤り$\vec{e}_i$が発生した場合、非零のシンドローム$\vec{s}$は誤りの位置$i$の\\
    2進数表現となっている。
  \end{frame}
  \begin{frame}{例}
    \begin{block}{例 7.34}
      先程挙げたパリティ検査行列で定義される$\code{H}_7$と同値な符号を用いる。\\
      置換$(1362547)$を例7.33で用いられた符号語$1101001$に適用すると、$\vec{u} = 1010101 \in \code{H}_7$を得る。 \\
      これが送信され$\vec{v} = 1010001$が受信された場合、\\
      シンドロームは$\vec{s} = \vec{v} H^T = 101$で、 これは$5$の2進数表現であり、\\
      よって$\vec{v}$の5番目を変更することで$\Delta(\vec{v}) = 1010101 = \vec{u}$を得る。
    \end{block}
  \end{frame}
  
  \begin{frame}{練習}
    \begin{block}{練習 7.6}
      例7.34で用いた$\code{H}_7$の同値な符号を用いる。\\
      $\vec{u} = 0111100$が送信され$\vec{v} = 0011100$が送信された場合、どうなるかを説明せよ。
    \end{block}
    \begin{block}{解答}
      板書する。
    \end{block}
  \end{frame}
  \begin{frame}{$q>2$の場合の完全符号}
    素数の類乗数$q > 2$に対しても似たような完全1重誤り訂正線形符号の構成が\\
    存在する。 $H$の列を
    \[ n = \frac{q^c - 1}{q - 1} = 1 + q + q^2 + \cdots + q^{c-1} \ \eqno{(7.5)} \]
    対の線形独立な長さ$c$の$\F_q$上のベクトルとなるようにとる。 \\
    (これは可能な最大数である。練習7.7を見よ) \\
    結果得られた線形符号は符号長$n$で次元$k=n-c$となり、\\
    最小距離$d= 3$よって$t=1$となる。 (再び練習7.7を見よ) \\
    2元の場合は、$R \rightarrow 1 \ (c \rightarrow \infty) $だが、$\text{Pr}_E \not\rightarrow 0$
  \end{frame}
  \begin{frame}{練習}
    \begin{block}{練習 7.7}
      $\vs{W} = \F_q^c$のとき、$\vs{W}$のどんな2つのベクトルの組も線形従属とならないような\\
      ベクトルの最大個数が$(q^c - 1) / (q - 1)$であることを示せ。\\
      そのようなベクトルの集合はどんなものもパリティ検査行列$H$を形成する場合、\\
      結果得られる$\F_q$上の線形符号は完全で1重誤り訂正となることを示せ。
    \end{block}
    \begin{block}{解答}
      $\vs{W}$は$c$個の零ベクトルのみを共通元にもつ1次元部分空間の直和と考えられ、\\
      含まれるベクトルの個数は$q^c$となっている。 \\
      このとき、$c$個の1次元部分空間から1つずつ非零な代表元をとってくることで、\\
      任意の2組が従属関係にないベクトルの最大個数の組を得られる。\\
      各部分空間の非零ベクトルの個数は$q-1$個なので、この最大個数は、\\
      $\vs{W}$の非零ベクトルの個数から$q-1$を割ることで得られる。\\
      つまり$(q^c - 1) / (q - 1)$となる。
    \end{block}
  \end{frame}

\end{document}
